%%\VignetteIndexEntry{Routines for fitting kinetic models with one or more state variables to chemical degradation data}
%%VignetteDepends{FME}
%%\usepackage{Sweave}
\documentclass[12pt,a4paper]{article}
\usepackage{a4wide}
%%\usepackage[lists,heads]{endfloat}
\usepackage{booktabs}
\usepackage{amsfonts}
\usepackage{latexsym}
\usepackage{amsmath}
\usepackage{amssymb}
\usepackage{graphicx}
\usepackage{parskip}
\usepackage[round]{natbib}
\usepackage{amstext}
\usepackage{hyperref}
\usepackage[latin1]{inputenc}

\newcommand{\Rpackage}[1]{{\normalfont\fontseries{b}\selectfont #1}}
\newcommand{\Robject}[1]{\texttt{#1}}
\newcommand{\Rclass}[1]{\textit{#1}}
\newcommand{\Rcmd}[1]{\texttt{#1}}

\newcommand{\RR}{\textsf{R}}

\RequirePackage[T1]{fontenc}
\RequirePackage{graphicx,ae,fancyvrb}
\IfFileExists{upquote.sty}{\RequirePackage{upquote}}{}
\usepackage{relsize}

\DefineVerbatimEnvironment{Sinput}{Verbatim}{baselinestretch=1.05}
\DefineVerbatimEnvironment{Soutput}{Verbatim}{fontfamily=courier,
                                              baselinestretch=1.05,
                                              fontshape=it,
                                              fontsize=\relsize{-1}}
\DefineVerbatimEnvironment{Scode}{Verbatim}{}  
\newenvironment{Schunk}{}{}

\hypersetup{  
  pdftitle = {mkin - Routines for fitting kinetic models with one or more state variables to chemical degradation data},
  pdfsubject = {Manuscript},
  pdfauthor = {Johannes Ranke},
  colorlinks = {true},
  linkcolor = {blue},
  citecolor = {blue},
  urlcolor = {red},
  hyperindex = {true},
  linktocpage = {true},
}

\begin{document}
\title{mkin -\\
Routines for fitting kinetic models with one or more state variables to chemical degradation data}
\author{\textbf{Johannes Ranke} \\
%EndAName
Product Safety \\
Harlan Laboratories Ltd. \\
Zelgliweg 1, CH--4452 Itingen, Switzerland}
\maketitle

\begin{abstract}
In the regulatory evaluation of chemical substances like plant protection
products (pesticides), biocides and other chemicals, degradation data play an
important role. For the evaluation of pesticide degradation experiments, 
detailed guidance has been developed, based on nonlinear optimisation. 
The \RR{} add-on package \Rpackage{mkin} implements fitting the models
recommended in this guidance from within R and calculates the recommended
statistical measures for data series within one or more compartments,
for parent and metabolites.
\end{abstract}


\thispagestyle{empty} \setcounter{page}{0}

\clearpage

\tableofcontents

\textbf{Key words}: Kinetics, FOCUS, nonlinear optimisation

\section{Introduction}
\label{intro}

Many approaches are possible regarding the evaluation of chemical degradation
data.  The \Rpackage{mkin} package \citep{pkg:mkin} in \RR{}
\citep{rcore2010} implements the approach recommended in the kinetics report
provided by the FOrum for Co-ordination of pesticide fate models and their
USe \citep{FOCUS2006} for simple data series for one parent compound in one
compartment.

\section{Example}
\label{exam}

In the following, requirements for data formatting are explained. Then the
procedure for fitting the four kinetic models recommended by the FOCUS group
to an example dataset for parent only given in the FOCUS kinetics report is
illustrated.  The explanations are kept rather verbose in order to lower the
barrier for \RR{} newcomers.

\subsection{Data format}

The following listing shows example dataset C from the FOCUS kinetics
report as distributed with the \Rpackage{kinfit} package

\begin{Schunk}
\begin{Sinput}
R> library("mkin")
R> FOCUS_2006_C
\end{Sinput}
\begin{Soutput}
    t parent
1   0   85.1
2   1   57.9
3   3   29.9
4   7   14.6
5  14    9.7
6  28    6.6
7  63    4.0
8  91    3.9
9 119    0.6
\end{Soutput}
\end{Schunk}

Note that the data needs to be in the format of a data frame containing
a variable \Robject{time} containing sampling times and a variable
\Robject{parent} containing the measured data. Replicate measurements are
not recorded in extra columns but simply appended, leading to multiple
occurrences of the sampling times \Robject{time}.

Small to medium size dataset can be conveniently entered directly as \RR{} code
as shown in the following listing

\begin{Schunk}
\begin{Sinput}
R> example_data <- data.frame(
+   time = c(0, 1, 3, 7, 14, 28, 63, 91, 119),
+   parent = c(85.1, 57.9, 29.9, 14.6, 9.7, 6.6, 4, 3.9, 0.6)
+ )
\end{Sinput}
\end{Schunk}



\bibliographystyle{plainnat}
\bibliography{references}

\end{document}
% vim: set foldmethod=syntax:
